\documentclass{article}
\usepackage[utf8]{inputenc}
\usepackage{authblk}

\title{Model of Charge Distribution in EMCCD}
\author[1]{Marie Yau}
\author[2]{Ivan Kotov}
\affil[1]{University of California, Berkeley}
\affil[2]{Brookhaven National Laboratory}
\date

\begin{document}

\maketitle

The Soft Inelastic X-ray Scattering (SIX) beamline at the National Synchrotron Light Source II located at the Brookhaven National Laboratory uses x-ray to study the composition of a material. It is done by letting the x-ray scatter over the material. The x-ray is then detected by an electron camera (EMCCD) that is composed of a pixelated grid. By measuring the charge accumulated in each pixel, the camera produces an image showing the proportion of the x-ray that was scattered over the material, enabling us to find out what its structure is. Using this method of x-ray scattering, it is possible to achieve a very high image accuracy because we can deduce the position of the x-ray with accuracy better than its pixel pitch. The goal of my project was to develop a program that calculates the exact position of the x-ray given an image. When we receive a charge distribution for an unknown original x-ray coordinate, we can fit two-dimensional Gaussian model to our distribution using the least squares method. We have estimated the model accuracy by precomputing charge values for many different x-ray positions on the detector using a mathematical formula. Since the x-ray generates a charge on the pixel it hits and its neighboring pixels, we generate a charge distribution over detector’s pixels for each x-ray position. As a result of my internship, I have developed a program that enables other scientists to study structures of various materials more accurately. During the process, I have improved my programming skills and learned to work with many optimization packages.

\end{document}
